\documentclass[11pt,a4paper]{article}
\RequirePackage{comment}
\RequirePackage{hyperref}

\newcommand{\xen}[1]{}
\newcommand{\xnen}[1]{#1}
\newcommand{\xkr}[1]{}
\newcommand{\xnkr}[1]{#1}
\newcommand{\xjp}[1]{}
\newcommand{\xnjp}[1]{#1}
\begin{xen}
  \renewcommand{\xen}[1]{#1}
  \renewcommand{\xnen}[1]{}
\end{xen}
\begin{xkr}
  \renewcommand{\xkr}[1]{#1}
  \renewcommand{\xnkr}[1]{}
\end{xkr}
\begin{xjp}
  \renewcommand{\xjp}[1]{#1}
  \renewcommand{\xnjp}[1]{}
\end{xjp}

\begin{document}
\title{Resume}
\author{염규선 | Gyusun Yeom | 廉圭先}
Seoul, Rep. of KOREA\\
\href{mailto:omniavinco@gmail.com}{omniavinco@gmail.com}\\
\url{https://www.linkedin.com/in/perlmint}\\
\url{https://github.com/perlmint}

\section{
  \xen{Techniques Summary}
  \xkr{기술 요약}
  \xjp{記述要約}
}
\begin{xen}
Generalist programmer. UI. Web server/client. Game logic. Cross-platform build managing.
\end{xen}
\begin{xkr}
범용 프로그래머. UI. 웹 서버/클라이언트. 게임 로직. 크로스 플랫폼 빌드 관리.
\end{xkr}
\begin{xjp}
プログラマ。UI。ウェブ サーバー/クリイアント。ゲームロジク。クロスプレッポム
\end{xjp}

\begin{xen}
Recently used languages are C\#, C++, Python, typescript/javascript, and golang. I have development experience with RDB(MySQL), NoSQL(Redis), version control(SVN, git, mercurial), some OS(Windows, Linux, OSX, AIX, Solaris, HP-UX). used RDB with ORM(SQL alchemy, Entity framework, sequelize) and raw SQL. I have written build script for some projects with gradle, CMake. And managed CI with Jenkins, Travis-CI. Little experience of Unity. Web application development experience with python - flask, typescript - react.js, angular.js 1 and express, C\# - asp.net, golang. Scripting - especially utility scripting - experience with bash, Powershell, Python, and Perl.
\end{xen}
\begin{xkr}
  최근에 사용한 언어는 C\#, C++, Python, Typescript/Javascript, 그리고 golang입니다. RDB(MySQL), NoSQL(Redis), 버전관리시스템(SVN, git, mercurial)을 사용한 경험이 있으며, 여러 운영체제(Windows, Linux, OSX, AIX, Solaris, HP-UX)에서 개발을 해본 경험이 있습니다. 그리고 RDB를 ORM(SQL alchemy, Entity framework, sequelize)을 사용하거나, SQL을 직접 작성하여 사용해보았습니다. 여러 프로젝트의 빌드 스크립트를 gradle, CMake로 작성한 경험이 있으며, Jenkins와 Travis-CI와 같은 CI툴을 직접 관리해보았습니다. 약간의 Unity경험과 python(flask), Typescript/Javascript(react.js, angular.js, express), C\#(asp.net), golang을 사용한 웹 어플리케이션 개발을 해보았습니다. 유틸리티 스크립트는 bash, Powershell, Python, Perl로 작성을 합니다.
\end{xkr}

\section{
  \xen{Education}
  \xkr{교육}
  \xjp{}
}
\subsection{
  \xen{Yonsei Univ.}
  \xkr{연세대학교}
  \xjp{延世大学}
}
\subsubsection{
  \xen{Computer Science Bachelor - 2017.2}
  \xkr{컴퓨터과학 학사 - 2017.2}
  \xjp{コンピューター科学 学士 - 2017.2}
}
GPA: 3.42/4.5
\subsubsection{
  \xen{Cognitive Science – 2017.2}
  \xkr{인지과학 연계전공 - 2017.2}
  \xjp{認知科学 連携専攻 - 2017.2}
}
\begin{xen}
  Psychology, Neuro biology, design.
\end{xen}
\begin{xkr}
  심리학, 신경생물학, 디자인
\end{xkr}
\begin{xjp}
  心理学、神経生物学、デザイン
\end{xjp}
\subsection{
  \xen{Software maestro}
  \xkr{소프트웨어 마에스트로}
  \xjp{ソフトウェア マエストロ}
}
\subsubsection{
  \xen{Essentials of Programming language}
  \xkr{프로그래밍 언어의 정수}
  \xjp{プログラミング言語の精髄}
}
\begin{xen}
  Learning programming language by implementing the interpreter for lisp-like language.
\end{xen}
\begin{xkr}
  Lisp과 비슷한 언어의 인터프리터를 구현하면서 프로그래밍 언어이론을 학습.
\end{xkr}
\begin{xjp}
  LISP見たいな言語のインタプリタを作る
\end{xjp}
\subsubsection{
  \xen{Mobile game development}
  \xkr{모바일 게임 개발}
  \xjp{モバイルゲーム開発}
}
\begin{xen}
  Developing mobile game from designing game.
\end{xen}
\begin{xkr}
  모바일 게임을 기획에서부터 시작해서 개발.
\end{xkr}
\begin{xjp}
  モバイルゲームを企画から始め、開発。
\end{xjp}

\section{
  \xen{Work experience}
  \xkr{업무 경력}
  \xjp{経歴}
}
\subsection{
  \xen{Synapsoft}
  \xkr{사이냅소프트}
}
2012.8\textasciitilde2014.4
\subsubsection{
  \xen{HTML Converter 2013}
  \xkr{HTML 변환기 2013}
  \xjp{HTML変換器2013}
}
\begin{xen}
  Program that convert office documents (doc/x, ppt/x, xls/x, hwp) to HTML document.
\end{xen}
\begin{xkr}
  doc/x, ppt/x, xls/x, hwp등의 오피스 문서를 HTML로 변환해주는 프로그램
\end{xkr}
\begin{xjp}
  docx/x, ppt/x, xls/x, hwpなどのオフィス文書をHTMLで変換するプログラム。
\end{xjp}

\begin{xen}
  Used C++
\end{xen}
\begin{xkr}
  C++ 사용
\end{xkr}
\begin{xjp}
  C++を使う。
\end{xjp}

\begin{xen}
  I developed skinning feature for HTML result and I18N, L10N support.
\end{xen}
\begin{xkr}
  HTML결과물의 스킨 기능과 I18N, L10N 지원 기능을 개발.
\end{xkr}
\begin{xjp}
  HTML変換結果物のスキン機能とI18N、L10Nサポート機能を開発。
\end{xjp}

\begin{xen}
  Skinning feature – Apply skin template for converted HTML result.
\end{xen}
\begin{xkr}
  스킨 기능 - HTML 변환 결과물에 스킨 템플릿을 적용
\end{xkr}

\begin{xen}
  I18N, L10N support – fix some encoding issue \& implement the gettext-like system, but language resource executable self-contained.
\end{xen}
\begin{xkr}
  I18M, L10N 지원 - 인코딩 문제 수정, GNU gettext와 비슷하지만 언어 리소스를 실행파일에 포함하는 시스템 구현.
\end{xkr}

\begin{xen}
  Refactoring codes – refactor old legacy codes for reducing code complexity.
\end{xen}
\begin{xkr}
  코드 리팩토링 - 코드 복잡도를 낮추기 위하여 오래된 레거시 코드를 리팩토링.
\end{xkr}

\begin{xen}
  Managing CI - multiple targets build test, unit test, and code quality check system.
\end{xen}
\begin{xkr}
  CI 관리 - 다중 타겟 빌드 테스트, 유닛 테스트, 코드 품질 측정 재구축
\end{xkr}

\subsubsection{
  \xen{Mobile office (Naver mobile office)}
  \xkr{모바일 오피스 (네이버 모바일 오피스)}
  \xjp{モバイルオフィス}
}
\begin{xen}
  Office document viewer \& editor.
\end{xen}
\begin{xkr}
  오피스 문서 뷰어 및 편집기
\end{xkr}

\begin{xen}
  Used C++.
\end{xen}
\begin{xkr}
  C++ 사용.
\end{xkr}

\begin{xen}
  I developed document importer. Some formats (doc, docx, ppt, hwp3) importer are developed by me.
\end{xen}
\begin{xkr}
  문서 임포터를 개발함. doc, docx, ppt, hwp3 포맷에 대응하는 임포터 개발.
\end{xkr}

\subsection{
  \xen{Mobilfactory}
  \xkr{모빌팩토리}
  \xjp{モビルファクトリー}
}
2014.4\textasciitilde2015.7
\subsubsection{
  \xen{ShootingHero}
  \xkr{슈팅히어로}
  \xjp{シュウチングヒーロ}
}
\begin{xen}
  Mobile flight shooting game.
\end{xen}
\begin{xkr}
  모바일 비행 슈팅 게임.
\end{xkr}

\begin{xen}
  I developed C\# server, Cocos2d-X client(C++), server deploying scripts to AWS, game server web operation tool, build scripts for android \& iOS, and integrated the game with publisher SDK.
\end{xen}
\begin{xkr}
  C\#서버, Cocos2d-X 클라이언트(C++), AWS로 서버 배포 스크립트, 게임서버 운영툴, 안드로이드 및 iOS 클라이언트 빌드스크립트, 퍼블리셔 SDK 연동을 개발 및 수행함.
\end{xkr}

\begin{xen}
  Server and client communicate via encrypted binary - serialized by thrift serializer - on TCP.  Powershell script that deploy server binary and config files to windows server on AWS EC2, S3. Also developed Game managing tool with ASP.net \& angular. I used JNI, Java, objective-C for integration.
\end{xen}
\begin{xkr}
  서버와 클라이언트는 thrift시리얼라이저에 의해 직렬화된 바이너리를 암호화 하여 TCP로 통신하도록 개발함. Powershell스크립트로 게임 서버 바이너리와 설정파일을 AWS EC2의 윈도우 서버로 S3를 활용하여 배포함. 또한 게임 운영툴은 ASP.net과 angular를 사용해서 개발함. SDK 연동을 위해서 JNI, Java, Objective-C를 사용함.
\end{xkr}

\section{
  \xen{Other personal projects}
  \xkr{기타 개인 프로젝트}
  \xjp{他のプロジェクト}
}
\subsection{Typedsequelize}
\url{https://github.com/perlmint/typedsequelize}

\begin{xen}
  Sequelize(JS ORM) code generator for typescript. Originally using sequelize with typescript need too many codes which represent same information. It generates that codes from the class definition.
\end{xen}
\begin{xkr}
  Typescript를 위한 Javascript ORM인 Sequelize 코드 생성기. Sequelize를 typescript에서 사용하기 위해서는 같은 정보를 의미하는 많은 코드를 작성해야한다. Typedsequelize는 그러한 코드를 클래스 정의로부터 생성한다.
\end{xkr}

\subsection{Goautoenv}
\url{https://github.com/Perlmint/goautoenv}

\begin{xen}
  Environment variable setter for golang. It supports environment setup per workspace. Also it inferences golang package name by repository url.
\end{xen}
\begin{xkr}
  golang을 위한 환경변수 설정 프로그램. 작업공간별로 다른 환경을 설정할 수 있게 해준다. 또한 저장소의 URL로부터 golang 패키지 이름을 추론하여 사용한다.
\end{xkr}

\subsection{glew-cmake}
\url{https://github.com/Perlmint/glew-cmake}

\begin{xen}
  Unofficial CMake support of glew(\url{https://github.com/nigels-com/glew}). It was created before official CMake support is added to glew. Automatically merge from main repository via CI even now.
\end{xen}
\begin{xkr}
  glew(\url{https://github.com/nigels-com/glew})의 비공식 CMake 지원. glew가 공식적으로 CMake를 지원하기 이전에 작성되었다. 지금까지도 CI에 의해서 glew 메인 저장소로부터 자동으로 머지를 수행되고있다.
\end{xkr}

\section{
  \xen{Awards}
  \xkr{수상}
  \xjp{}
}
\subsection{
  \xen{NOS season2. 2nd prize}
  \xkr{NOS 시즌2. 2등상 수상}
}
\begin{xen}
  I won 2nd prize at NOS(Nexon Open Studio) season 2. My team created iOS game with iOS APIs(Quatz Core, Core animation).
\end{xen}
\begin{xkr}
  NOS(Nexon Open Studio) 시즌2에서 2등상을 수상했다. iOS게임을 iOS제공 API(Quartz Core, Core animation)등을 사용하여 개발하였다.
\end{xkr}

\section{
  \xen{Language Ability}
  \xkr{언어 능력}
  \xjp{言語能力}
}
\subsection{
  \xen{Korean}
  \xkr{한국어}
  \xjp{韓国語}
}
\begin{xen}
  Native
\end{xen}
\begin{xkr}
  모국어
\end{xkr}
\begin{xjp}
  母語
\end{xjp}

\subsection{
  \xen{English}
  \xkr{영어}
  \xjp{英語}
}
\begin{xen}
  Intermediate.
\end{xen}
\begin{xkr}
  중급
\end{xkr}
\begin{xjp}
  中級
\end{xjp}

\subsection{
  \xen{Japanese}
  \xkr{일본어}
  \xjp{日本語}
}
\begin{xen}
  Intermediate.
\end{xen}
\begin{xkr}
  중급
\end{xkr}
\begin{xjp}
  中級
\end{xjp}
\\JLPT N2

\begin{flushright}
\xnen{\href{./resume-en.html}{English}}
\hspace{1em}
\xnkr{\href{./resume-kr.html}{한국어}}
\hspace{1em}
\xnjp{\href{./resume-jp.html}{日本語}}
\end{flushright}
\end{document}
